% This is the file for exercise 1e in Project 1

A list of what we are supposed to do from the exercise text:

\begin{itemize}

\item Compare your results with those from the LU decomposition codes for the matrix of sizes $10\times 10$, $100\times 100$ and
$1000\times 1000$ (use the library functions provided  on the webpage of the course. Alternatively, if you use armadillo as a library, you can use the similar function for LU decomposition.  The armadillo function for the LU decomposition is called $LU$ while the function for solving linear sets of equations is called $solve$.
Use for example the unix function \emph{time} when you run your codes and compare the time usage between LU decomposition and your tridiagonal solver. Alternatively, you can use the functions in C++, Fortran or Python that measure the time used)

\item Make a table of the results and comment the differencesin execution time.

\item How many floating point operations does the LU decomposition use to solve the set of linear equations?

\item Answer: Can you run the standard LU decomposition
for a matrix of the size $10^5\times 10^5$?

\item Comment your results.

To compute the elapsed time in c++ you can use the following statements (see comment)
%\begin{print}
%...
%#include "time.h"   //  you have to include the time.h header
%int main()
%{
%    // declarations of variables 
%    ...
%    clock_t start, finish;  //  declare start and final time
%    start = clock();
%    // your code is here, do something and then get final time
%    finish = clock();
%    ( (finish - start)/CLOCKS_PER_SEC );
%...
%\end{print}
\end{itemize}
