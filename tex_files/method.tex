\begin{itemize}

\item Algorithm

\item Theoretical models and technicalities

\end{itemize}

\subsection{Algorithm}

The first algorithms that we used are based on Gaussian elimination. This method starts by making the matrix upper triangular, thereafter one can solve the set of linear equations by starting at the bottom of the matrix where there are only one unknown. 

Our matrix is a tridiagonal matrix and therefore, just a few matrix operations were necessary to make the matrix upper triangular. Those are called forward substitution. The process of solving the set of equation from the bottom is called backward substitution.

\subsection{The general tridiagonal matrix}
The algorithm is first written for an equation with a general tridiagonal matrix on the short form:
\[
\mathbf{A}\mathbf{v}=\mathbf{f}
\]
The full form looks like this:
\[ \frac{1}{h^2}
    \begin{bmatrix}
    	b_1& c_1& 0 &\cdots & 0 &0 \\
        a_1 & b_2 & c_2 &0 &\cdots &0 \\
        0&a_2 &b_3 & c_3 & \cdots & 0 \\
        \vdots& \vdots & \ddots &\ddots &\ddots & \vdots \\
        0&0 & \cdots &a_{n-2} &b_{n-1}& c_{n-1} \\
        0&0 & \cdots & 0  &a_{n-1} & b_n \\
        \end{bmatrix}
\begin{bmatrix}
	v_1\\
	v_2\\
	v_3\\
	\vdots\\
	v_{n-1}\\
	v_{n}\\
\end{bmatrix}=
\begin{bmatrix}
	f_1\\
	f_2\\
	f_3\\
	\vdots\\
	f_{n-1}\\
	f_{n}\\
\end{bmatrix}
\]

The method used to solve the equation numerically was to first to a forward substitution to obtain a upper triangular matrix and then a backwards substitution to obtain the solution.

The algorithm preformed on the matrix values was:

\paragraph{The forward substitution:\hspace{4cm}}

\hspace{1cm}\linebreak
The values on the diagonal:
\[
\tilde{b}_i =  b_i - \left(\frac{a_i \cdot c_i}{\tilde{b}_{i-1}}\right)\linebreak
\]
The values on the right side of the equation:
\[
\tilde{f}_i =  f_i - \left(\frac{\tilde{f}_{i-1} \cdot a_i}{\tilde{b}_{i-1}}\right)
\]
for $i = 2, \dots, n $

\paragraph{The backward substitution:\hspace{4cm}}

\hspace{1cm}\linebreak
\[
v_i = \frac{\left(\tilde{f}_i - c_i\cdot v_{i+1}\right)}{\tilde{b}_i}
\]
for $i = n-1, \dots, 1 $. The boundary values are known ($v_0 = v_{n+1} = 0$). $v_{n}$ was calculated first and like this:
\[
v_{n} = \frac{\tilde{f}_{n}}{\tilde{b}_{n}}
\]

\subsection{This specific tridiagonal matrix}

The matrix in our case had the same values on the different diagonals:
\begin{multicols}{3}

\[
a_i = -1
\]

\[
b_i = 2
\]

\[
c_i = -1
\]

\end{multicols}

We can then pre-calculate some of the operations in the algorithm and if becomes then:

\paragraph{The forward substitution:\hspace{4cm}}

\hspace{1cm}\linebreak
The values on the diagonal:
\[
\tilde{b_i} =  b_i - \left(\frac{a_i \cdot c_i}{\tilde{b}_{i-1}}\right) = 2 - \left(\frac{(-1) \cdot (-1)}{\tilde{b}_{i-1}}\right) = 
\]
\[
2 - \frac{1}{\tilde{b}_{i-1}} =  \frac{i + 1}{i}
\]

The last expression can be shown to be true, by calculating some of the first values:
\[
\tilde{b}_2 = 2 - \frac{1}{2} = \frac{3}{2} = \frac{2 + 1}{2}
\]
\[
\tilde{b}_3 = 2 - \frac{1}{\frac{3}{2}} = 2 - \frac{2}{3} = \frac{12}{6} - \frac{4}{6} = \frac{8}{6} = \frac{4}{3} = \frac{3+1}{3}
\]

The values on the right side of the equation:
\[
\tilde{b_i} =  b_i - \left(\frac{\tilde{f}_{i-1} \cdot c_i}{\tilde{b}_{i-1}}\right) = b_i - \left(\frac{\tilde{f}_{i-1} \cdot (-1)}{\tilde{b}_{i-1}}\right) = b_i + \left(\frac{\tilde{f}_{i-1}}{\tilde{b}_{i-1}}\right)
\]

If we put in $\tilde{b}_{i-1} = \frac{i}{i-1}$:

\[
\tilde{f}_i = f_i + \frac{(i-1)\tilde{f}_{i-1}}{i}
\]

for $i = 2,\, \dots\, n $.

\paragraph{The backward substitution:\hspace{4cm}}

\hspace{1cm}\linebreak
\[
v_i = \frac{\left(\tilde{f}_i + v_{i+1}\right)}{\tilde{b}_i}
\]
for $i = n-1, \dots, 1 $. The boundary values are known ($v_0 = v_{n+1} = 0$). $v_{n}$ was calculated first and like this:
\[
v_{n} = \frac{\tilde{f}_{n}}{\tilde{b}_{n}}
\]

If we put in $\tilde{b}_{i-1} = \frac{i}{i-1}$:

\[
v_{i} = \frac{(i-1)(\tilde{f}_i + v_{i+1})}{i}
\]

\[
v_n = \frac{(i-1)\tilde{f}_n}{i}
\]


\subsection{LU decomposition}


The last method that we tried out is called LU decomposition. When using this method we factorized our matrix into two other matrices, one a lower triangular matrix and the other a upper triangular matrix. Here is an example for a $ 4 \cross 4$-matrix:

\[
A = LU = 
    \begin{bmatrix}
    	1& 0& 0 & 0 \\
        l_{21} & 1 & 0 &0\\
        l_{31}&l_{32} & 1 & 0 \\
        l_{41} & l_{42} & l_{43} & 1\\
        \end{bmatrix}
    \begin{bmatrix}
    	u_{11}& u_{12}& u_{13} &u_{14} \\
        0 & u_{22} & u_{23} &u_{24}\\
        0& 0  &u_{33} & u_{34} \\
        0&0 & 0 & u_{44}\\
        \end{bmatrix}
\]

LU decomposition exists it the determinant of the matrix is non-zero.

The method goes like this:

\[
\textbf{A}\textbf{v} = \textbf{f} \implies \textbf{L}\textbf{U}\textbf{v} = \textbf{f}
\]
\[
\textbf{U}\textbf{v} = \textbf{L}^{-1}\textbf{f} = \textbf{w}
\]

We first solve:
\[
\textbf{L}^{-1}\textbf{f} = \textbf{w}
\]

Then we can easily get $\textbf{x}$, because of the shape of $\textbf{U}$, from:
\[
\textbf{Ux}=\textbf{w} \implies 
\begin{bmatrix}
		u_{11}& u_{12}& u_{13} &u_{14} \\
        0 & u_{22} & u_{23} &u_{24}\\
        0& 0  &u_{33} & u_{34} \\
        0&0 & 0 & u_{44}\\
        \end{bmatrix}
\begin{bmatrix}
x_1\\
x_2\\
x_3\\
x_4\\
\end{bmatrix} =
\begin{bmatrix}
w_1\\
w_2\\
w_3\\
w_4\\
\end{bmatrix}        
\]

The bottom equation has only one unknown, so we start there.