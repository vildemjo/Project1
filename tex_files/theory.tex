\begin{equation}\label{eq:secderiv}
   -\frac{v_{i+1}+v_{i-1}-2v_i}{h^2} = f_i  \hspace{0.5cm} \mathrm{for} \hspace{0.1cm} i=1,\dots, n,
\end{equation}
where $f_i=f(x_i)$.

Rewriting the Equation \ref{eq:secderiv} as a linear set of equations of the form

\begin{equation*}
   \mathbf{A}\mathbf{v} = \tilde{\mathbf{b}},
\end{equation*}
where $\mathbf{A}$ is an $n\times n$  tridiagonal matrix which we rewrite as

\[
    \mathbf{A} = \begin{bmatrix}
    	2& -1& 0 &\cdots & 0 &0 \\
        -1 & 2 & -1 &0 &\cdots &0 \\
        0&-1 &2 & -1 & \cdots & 0 \\
        \vdots& \vdots & \ddots &\ddots &\ddots & \vdots \\
        0&0 & \cdots &-1 &2& -1 \\
        0&0 & \cdots & 0  &-1 & 2 \\
        \end{bmatrix},
\]
and $\tilde{b}_i=h^2f_i$.\\

Writing out the equations:
\begin{align*}
i = 1: &-v_0+2v_1-v_2 = f_1 h^2\\
i = 2: &-v_1+2v_2-v_3 = f_2 h^2\\
\vdots \indent &\indent \vdots\\
i = n: &-v_{n-1}+2v_n-v_{n+1} = f_n h^2\\
\end{align*}

where $v_0 = 0$ and $v_{n+1} = 0$.

Adding some zeros to the equations for illustrative purposes:

\begin{align*}
&i = 1:  &2v_1-v_2 + 0 + 0 + 0 +0 + \cdots + 0 = f_1 h^2\\
&i = 2: &-v_1+2v_2-v_3 +0 + 0 + \cdots + 0 = f_2 h^2\\
&i = 3:  &0 -v_2+2v_3-v_4 + 0 + \cdots + 0 = f_3 h^2\\
&\indent\vdots \indent &\vdots \indent \indent \indent \\
&i = n-1: &0 + \cdots+ 0 -v_{n-2}+2v_{n-1} - v_n= f_{n-1} h^2\\
&i = n: &0 + \cdots + 0 + 0 + 0 -v_{n-1}+2v_n= f_n h^2\\
\end{align*}

We can now recognize the equations as a product between a matrix an a vector that equals another vector.\\

Defining vectors that fit:
\begin{multicols}{3}
$$\textbf{v} = \begin{bmatrix}
	v_1\\
	v_2\\
	\vdots\\
	v_{n}\\
\end{bmatrix}$$

$$ \tilde{\textbf{b}} = \begin{bmatrix}
	f_1 h^2\\
	f_2 h^2\\
	\vdots\\
	f_{n} h^2\\
\end{bmatrix} $$
\end{multicols}

The matrix is the matrix $\textbf{A}$. Then we have:
\[
    \begin{bmatrix}
    	2& -1& 0 &\cdots & 0 &0 \\
        -1 & 2 & -1 &0 &\cdots &0 \\
        0&-1 &2 & -1 & \cdots & 0 \\
        \vdots& \vdots & \ddots &\ddots &\ddots & \vdots \\
        0&0 & \cdots &-1 &2& -1 \\
        0&0 & \cdots & 0  &-1 & 2 \\
        \end{bmatrix}
\begin{bmatrix}
	v_1\\
	v_2\\
	v_3\\
	\vdots\\
	v_{n-1}\\
	v_{n}\\
\end{bmatrix}=
\begin{bmatrix}
	f_1 h^2\\
	f_2 h^2\\
	f_3 h^2\\
	\vdots\\
	f_{n-1} h^2\\
	f_{n} h^2\\
\end{bmatrix}
\implies \mathbf{A}\mathbf{v}=\tilde{\mathbf{b}}
\]

Putting the solution into the Poisson equation:

The Poisson equation:
\[ -u''(x) = f(x)\]

In this case $f(x) = 100 e^{-10x}$.\\

The solution and derivatives:
\begin{align*}
u(x) &= 1 - \left(1-e^{-10}\right) x - e^{-10x}\\
u'(x) &= \left(1-e^{-10}\right)+ 10 e^{-10x}\\ 
u''(x) &= -100 e^{-10x}\\
\end{align*}

The solution is a solution of the Poisson equation:

\[ -u''(x) = -(-100) e^{-10x} = 100 e^{-10x} = f(x) \]
